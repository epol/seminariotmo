\documentclass{beamer}

\usetheme{metropolis}

\metroset{sectionpage=none,progressbar=frametitle,subsectionpage=progressbar}
\metroset{numbering=fraction}
\metroset{block=fill}

%\usepackage{pgfpages}
%\setbeameroption{show notes on second screen}

\usepackage{appendixnumberbeamer}

\usepackage[utf8x]{inputenc}
\usepackage[italian]{babel}

\usepackage{amsmath}
\usepackage{amssymb}
\usepackage{amsthm}
\usepackage{xfrac}
\usepackage{mathtools}
\usepackage{graphicx}
\usepackage{hyperref}
\usepackage{mathtools}
\usepackage{xcolor}

\usepackage{tikz}
\usetikzlibrary{arrows}

% to do small font tabular columns
%\usepackage{array}


\newcounter{counter1}

\theoremstyle{plain}
\newtheorem{myteo}[counter1]{Teorema}
\newtheorem{mylem}[counter1]{Lemma}
\newtheorem{mypro}[counter1]{Proposizione}
\newtheorem{mycor}[counter1]{Corollario}
%\newtheorem*{myteo*}{Teorema}
%\newtheorem*{mylem*}{Lemma}
%\newtheorem*{mypro*}{Proposizione}
%\newtheorem*{mycor*}{Corollario}

\theoremstyle{definition}
\newtheorem{mydef}[counter1]{Definizione}
\newtheorem{myes}[counter1]{Esempio}
%\newtheorem{myex}[counter1]{Esercizio}
%\newtheorem*{mydef*}{Definizione}
%\newtheorem*{myes*}{Esempio}
%\newtheorem*{myex*}{Esercizio}

\theoremstyle{remark}
%\newtheorem{mynot}[counter1]{Nota}
\newtheorem{myoss}[counter1]{Osservazione}
%\newtheorem*{mynot*}{Nota}
%\newtheorem*{myoss*}{Osservazione}

\newcommand{\obar}[1]{\overline{#1}}
\newcommand{\ubar}[1]{\underline{#1}}

\newcommand{\set}[1]{\left\{#1\right\}}
\newcommand{\pa}[1]{\left(#1\right)}
\newcommand{\ang}[1]{\left<#1\right>}
\newcommand{\bra}[1]{\left[#1\right]}
\newcommand{\abs}[1]{\left|#1\right|}
\newcommand{\norm}[1]{\left\|#1\right\|}
\newcommand{\ceil}[1]{\left\lceil#1\right\rceil}
\newcommand{\floor}[1]{\left\lfloor#1\right\rfloor}

\newcommand{\pfrac}[2]{\pa{\frac{#1}{#2}}}
\newcommand{\bfrac}[2]{\bra{\frac{#1}{#2}}}
\newcommand{\psfrac}[2]{\pa{\sfrac{#1}{#2}}}
\newcommand{\bsfrac}[2]{\bra{\sfrac{#1}{#2}}}

\newcommand{\der}[2]{\frac{\partial #1}{\partial #2}}
\newcommand{\pder}[2]{\pfrac{\partial #1}{\partial #2}}
\newcommand{\sder}[2]{\sfrac{\partial #1}{\partial #2}}
\newcommand{\psder}[2]{\psfrac{\partial #1}{\partial #2}}

\newcommand{\intl}{\int \limits}

\DeclareMathOperator{\de}{d}
\DeclareMathOperator{\id}{Id}
\DeclareMathOperator{\len}{len}

\DeclareMathOperator{\gl}{GL}
\DeclareMathOperator{\aff}{Aff}
\DeclareMathOperator{\isom}{Isom}

\DeclareMathOperator{\im}{Im}
\DeclareMathOperator{\re}{Re}
\DeclareMathOperator{\sign}{sign}




\title{Problemi di minimo su reti di comunicazioni e teoria dei giochi}
\date{\today}
\author{Enrico Polesel}
%\institute{Universit\`a di Pisa}


\begin{document}
\maketitle



\section{Introduzione e instradamento socialmente ottimo}

\subsection{Definizioni ed esempi}

\begin{frame}{Esempi}
  Reti:
  \begin{itemize}
  \item Reti stradali,
%  \item reti di distribuzione di risorse (e.g. gas),
  \item reti di telecomunicazione (internet).
  \end{itemize}

  Costi:
  \begin{itemize}
  \item Tempo di percorrenza (latenza),
  \item perdita di pacchetti,
  \item (costo economico).
  \end{itemize}

  Tipi di comportamento:
  \begin{itemize}
  \item Socialmente ottimo,
  \item anarchico,
  \item parzialmente ottimo.
  \end{itemize}
\end{frame}

\begin{frame}{Modello}
  \begin{center}
    \begin{tikzpicture}[mnode/.style={circle,fill=blue!20},>=latex,snode/.style={circle,fill=red!70}]
      \node[mnode] (1) at (0,0) {1};
      \node[mnode] (2) at (2,2) {2};
      \node[mnode] (3) at (4,2) {3};
      \node[mnode] (4) at (3,-2) {4};
      \node[mnode] (5) at (6,0) {5};
      \draw [->] (1) to [bend left] (2);
      \onslide<3->{\node at (1,1) {$l_1(x_1)$};}
      \draw [->] (2) to [bend right] (3);
      \onslide<3->{\node at (3,2.7) {$l_2(x_2)$};}
      \draw [->] (3) to [bend right] (2);
      \onslide<3->{\node at (3,1.3) {$l_3(x_3)$};}
      \draw [->] (3) to [bend left] (5);
      \onslide<3->{\node at (5.5,1.7) {$l_4(x_4)$};}
      \draw [->] (1) to [bend right] (4);
      \onslide<3->{\node at (0.6,-1.7) {$l_5(x_5)$};}
      \draw [->] (3) -- (4);
      \onslide<3->{\node at (4,0) {$l_6(x_6)$};}
      \draw [->] (4) -- (2);
      \onslide<3->{\node at (2,0) {$l_7(x_7)$};}
      \draw [->] (4) to [bend right] (5);
      \onslide<3->{\node at (5,-1) {$l_8(x_8)$};}

      \onslide<2->{
        \node[snode] (S1) at (-2,0) {S1};
        \node[snode] (S2) at (-2,2) {S2};
        \node[snode] (T1) at (8,0) {T1};
        \node[snode] (T2) at (8,-2) {T2};
        \draw[->] (S1) -- node[above]{traffico 1} (1);
        \draw[->] (5) -- node[above]{traffico 1} (T1);
        \draw[->] (S2) -- node[above]{traffico 2} (2);
        \draw[->] (4) -- node[below]{traffico 2} (T2);
      }
    \end{tikzpicture}
  \end{center}
  \begin{columns}[T]
    \begin{column}{.6\textwidth}
      \begin{itemize}
      \item<1-> Grafo diretto $(V,A)$,
      \item<2-> traffico (infinitamente divisibile),
      \end{itemize}
    \end{column}
    \begin{column}{.35\textwidth}
      \begin{itemize}
      \item<3-> costi.
      \end{itemize}
    \end{column}
  \end{columns}
\end{frame}

\begin{frame}{Istanza}
  Un'instanza di instradamento (\textit{routing instance}) è
  $R=(V,A,P,s,t,X,l)$ con:
  \begin{itemize}
  \item $(V,A)$ il grafo diretto,
  \item $s = \bra{s_i}_{i\in W}$ vettore delle sorgenti (per un opportuno $W$),
  \item $t = \bra{t_i}_{i\in W}$ vettore delle destinazioni,
  \item $P = \cup _{i\in W} P_i$ con $P_i$ insieme dei
    percorsi da $s_i$ a $t_i$,
  \item $X = \bra{X_i}_{i\in W}$ dove $X_i$ rappresenta la quantità di
    traffico da instradare da $s_i$ a $t_i$,
  \item $l = \bra{l_j}_{j\in A}$ dove $l_j(x_j)$ è il costo del link
    $j$ con un traffico $x_j$.
  \end{itemize}
  
  Indichiamo con $\mathcal{R}$ l'insieme di tutte le istanze, con
  $\mathcal{R}^{aff},\mathcal{R}^{conv},$ i sottoinsiemi delle istanze con
  le $l_j$ rispettivamente affine e convesse.

  Supponiamo che le $l_j$ siano non negative e non decrescenti.
\end{frame}

\begin{frame}{Soluzione di un'instanza e suo costo}
  Data un'instanza $R=(V,A,P,s,t,X,l)$ un vettore
  $x = \bra{x_j} \cup \bra{x_p}$ è soluzione se:
  \begin{itemize}
  \item $\displaystyle \forall p\in P\;\; x_p \ge 0$,
  \item $\displaystyle \forall j\in A\;\; x_j = \sum_{p\in P,p\ni j} x_p$,
  \item $\displaystyle \forall i\in W\;\; \sum_{p\in P_i} x_p = X_i$.
  \end{itemize}
  
  Il costo (globale) della soluzione sarà dato da:
  \[ C(x) = \sum_{j\in A} l_j(x_j)x_j \]
\end{frame}

\subsection{Instradamento socialmente ottimo}

\begin{frame}{Instradamento socialmente ottimo}
  \begin{mydef}[Instradamento socialmente ottimo]
    Diciamo che una soluzione $x$ di $R \in \mathcal{R}$ è
    \textbf{socialmente ottima} se minimizza la funzione costo
    $C(x) = \sum_{j\in A} l_j(x_j)x_j$, indichiamo una tale soluzione
    con $x^{SO}(R)$.
  \end{mydef}
  \[
    \begin{matrix}
      \text{minimizzare} & \sum _{j\in A} x_j l_j(x_j) \\
      \text{tale che} &   x_p \ge 0 & p\in P \\
      & x_j = \sum_{p\in P,p\ni j} x_p & j\in A \\
      &\sum_{p\in P_i} x_p = X_i & i\in W
    \end{matrix}
  \]
  \begin{myoss}
    I vincoli sono tutti lineari.
  \end{myoss}
  \begin{myoss}
    La soluzione potrebbe non essere unica.
  \end{myoss}
\end{frame}

\begin{frame}{Esempio di Pigou}
  \begin{center}
    \begin{tikzpicture}[mnode/.style={circle,fill=blue!20,minimum size=0.65cm},>=latex,snode/.style={circle,fill=red!70}]
      \node[mnode] (1) at (0,0) {s}; 
      \node[mnode] (2) at (3,0) {t};
      \draw [->] (1) to [bend left] node[above] {$l_1(x) = x$} (2);
      \draw [->] (1) to [bend right] node[below] {$l_2(x) = 1$} (2);
      \draw [->,color=red] (-4,0) -- node[above] {1 unità di traffico} (1) ;
      \draw [->,color=red] (2) --  (5,0) ;
      \node at (7,0) {};
    \end{tikzpicture}
  \end{center}
  La soluzione ottima è $x_1 = x_2 = \frac{1}{2}$ con costo
  \[ C(x) = \frac{1}{4} + \frac{1}{2} = \frac{3}{4} \]
  Metà del traffico arriverà con una latenza $\frac{1}{2}$ e metà con
  latenza $1$.
\end{frame}

\section{Equilibrio di Wardrop}

\subsection{Teoria dei giochi}

\begin{frame}{Definizione}
  \begin{mydef}[Gioco strategico]
    Un gioco è dato da $\pa{\mathcal{I},\pa{S_i}_{i\in\mathcal{I}},
      \pa{u_i}_{i\in\mathcal{I}}}$ con:
    \begin{itemize}
    \item $\mathcal{I}$ insieme \textbf{finito} dei giocatori,
    \item $S_i$ insieme non vuoto delle azioni possibili per il
      giocatore $i$,
    \item $u_i: \prod _{i\in\mathcal{I}} S_i \to \mathbb{R}$ guadagno
      per il giocatore $i$ in funzione delle azioni di tutti i
      giocatori.
    \end{itemize}
  \end{mydef}
  L'obiettivo di ogni giocatore è di massimizzare il proprio guadagno.

  Definiamo:
  \begin{itemize}
  \item $S = \prod _{i\in\mathcal{I}} S_i$,
  \item se $s = \bra{s_i}_{i\in\mathcal{I}} \in S$ allora $s_{-i} =
    \bra{s_j}_{j\neq i}$
  \item $\displaystyle S_{-i} = \prod _{j\in\mathcal{I},j\neq i} S_i$,
  \end{itemize}
\end{frame}

\begin{frame}{Esempio}
  In \textbf{matching pennies} ogni giocatore sceglie se mostrare una
  moneta con testa o croce, se le monete sono uguali vince il primo
  giocatore, se sono diverse il secondo.
  
  \begin{columns}[T]
    \begin{column}{0.4\textwidth}
      \[ \mathcal{I} = \set{1,2} \]
      \[ S_1 = S_2 = \set{\text{Testa},\text{Croce}} \]
    \end{column}
    \begin{column}{0.4\textwidth}
      \begin{tabular}{rcc}
        & Testa & Croce \\
        \cline{2-3}
        Testa & \multicolumn{1}{|c|}{$1,-1$} & \multicolumn{1}{|c|}{$-1,1$}  \\
        \cline{2-3}
        Croce & \multicolumn{1}{|c|}{$-1,1$} & \multicolumn{1}{|c|}{$1,-1$}  \\
        \cline{2-3}
      \end{tabular}
      \[ u_1,u_2 \]
    \end{column}
  \end{columns}
\end{frame}

\begin{frame}{Strategie}
  \begin{mydef}[Strategia dominante]
    Per un giocatore $i\in\mathcal{I}$ un'azione $s_i\in S_i$ è una
    strategia dominante se:
    \[ \forall s_{-i} \in S_{-i}\;\; \forall s_i' \in S_i\;\;\;
      u_i\pa{s_i,s_{-i}} \ge u_i\pa{s_i',s_{-i}} \]
  \end{mydef}

  Ci aspettiamo che un giocatore razionale giochi una strategia
  dominante, se esiste.
\end{frame}

\begin{frame}{Dilemma del prigioniero}
  Due criminali vengono arrestati per aver commesso un reato, hanno a
  disposizione due scelte: collaborare o non collaborare con la
  polizia.

  \begin{tabular}{rcc}
    & Non collaborare & Collaborare \\
    \cline{2-3}
    Non collaborare & \multicolumn{1}{|c|}{$-2,-2$} & \multicolumn{1}{|c|}{$-5,-1$}  \\
    \cline{2-3}
    Collaborare & \multicolumn{1}{|c|}{$-1,-5$} & \multicolumn{1}{|c|}{$-4,-4$}  \\
    \cline{2-3}
  \end{tabular}

  \begin{block}{Strategia dominante}
    Per entrambi la strategia ``collaborare'' è una strategia dominante.
  \end{block}
\end{frame}

\begin{frame}{Equilibrio di Nash}
  Sia  $\pa{\mathcal{I},\pa{S_i}_{i\in\mathcal{I}},
    \pa{u_i}_{i\in\mathcal{I}}}$ un gioco.
  \begin{mydef}[Equilibrio di Nash]
    $s^* \in S$ è un equilibrio di Nash se
    \[ \forall i\in \mathcal{I}\;\; \forall s_i \in S_i \;\;\;
      u_i\pa{s^*_i,s^*_{-i}} \ge u_i\pa{s_i,s^*_{-i}} \]
  \end{mydef}

  Nessun giocatore è interessato a spostarsi da un equilibrio di Nash.
\end{frame}

\begin{frame}{Dilemma del prigioniero con suicidio}
    \begin{tabular}{rccc}
    & Non collaborare & Collaborare & Suicidarsi \\
    \cline{2-4}
    Non collaborare & \multicolumn{1}{|c|}{$-2,-2$} & \multicolumn{1}{|c|}{$-5,-1$} & \multicolumn{1}{|c|}{$0,-20$}  \\
    \cline{2-4}
    Collaborare & \multicolumn{1}{|c|}{$-1,-5$} & \multicolumn{1}{|c|}{$-4,-4$}  & \multicolumn{1}{|c|}{$-4,-20$} \\
    \cline{2-4}
    Suicidarsi & \multicolumn{1}{|c|}{$-20,0$} & \multicolumn{1}{|c|}{$-20,-4$}  & \multicolumn{1}{|c|}{$-20,-20$} \\
    \cline{2-4}
  \end{tabular}
  \vfill
  
  In questo caso non c'è una strategia dominante.

  \begin{block}{Equilibrio di Nash}
    $\pa{\text{Collaborare},\text{Collaborare}}$ è un equilibrio di Nash.
  \end{block}
\end{frame}

\begin{frame}{Esempi}
  \begin{columns}[T]
    \begin{column}{0.45\textwidth}
      \begin{tabular}{rcc}
        & Testa & Croce \\
        \cline{2-3}
        Testa & \multicolumn{1}{|c|}{$1,-1$} & \multicolumn{1}{|c|}{$-1,1$}  \\
        \cline{2-3}
        Croce & \multicolumn{1}{|c|}{$-1,1$} & \multicolumn{1}{|c|}{$1,-1$}  \\
        \cline{2-3}
      \end{tabular}
    \end{column}
    \begin{column}{0.45\textwidth}
      In \textbf{matching pennies} non esiste alcun equilibrio di Nash.
    \end{column}
  \end{columns}
  \vfill
  
  \begin{columns}[T]
    \begin{column}{0.45\textwidth}
      \begin{tabular}{rcc}
        & Opera & Football \\
        \cline{2-3}
        Opera & \multicolumn{1}{|c|}{$2,1$} & \multicolumn{1}{|c|}{$0,0$}  \\
        \cline{2-3}
        Football & \multicolumn{1}{|c|}{$0,0$} & \multicolumn{1}{|c|}{$1,2$}  \\
        \cline{2-3}
      \end{tabular}
    \end{column}
    \begin{column}{0.45\textwidth}
      Nella \textbf{battaglia dei sessi} esistono due equilibri di
      Nash: $\pa{\text{Opera},\text{Opera}}$ e
      $\pa{\text{Football},\text{Football}}$.
    \end{column}
  \end{columns}
\end{frame}

\subsection{Equilibrio di Wardrop}

\begin{frame}{Reti e utenti}
  Rivediamo l'esempio di Pigou dopo averlo ``discretizzato'' a due giocatori:
  \begin{center}
    \begin{tikzpicture}[mnode/.style={circle,fill=blue!20,minimum size=0.65cm},>=latex,snode/.style={circle,fill=red!70}]
      \node[mnode] (1) at (0,0) {s}; 
      \node[mnode] (2) at (3,0) {t};
      \draw [->] (1) to [bend left] node[above] {$l_1(x) = x/4$}
      node[below] {\small \textcolor{green}{$1$} \textcolor{blue}{$2$}} (2);
      \draw [->] (1) to [bend right] node[below] {$l_2(x) = 1/2$}
      node[above] {\small \textcolor{green}{$1$} \textcolor{blue}{$0$}}(2);
      \draw [->,color=red] (-4,0) -- node[above] {$2$ unità di traffico} (1) ;
      \draw [->,color=red] (2) --  (5,0) ;
      \node at (7,0) {};
    \end{tikzpicture}
  \end{center}
  
  \begin{columns}
    \begin{column}{0.56\textwidth}
      \begin{tabular}{rcc}
        & Link $1$ & Link $2$ \\
        \cline{2-3}
        Link $1$ & \multicolumn{1}{|c|}{$0.5,0.5$} & \multicolumn{1}{|c|}{$0.25,0.5$}  \\
        \cline{2-3}
        Link $2$ & \multicolumn{1}{|c|}{$0.5,0.25$} & \multicolumn{1}{|c|}{$0.5,0.5$}  \\
        \cline{2-3}
      \end{tabular}
      \vspace{3px}

      Strategia dominante: $\pa{\text{Link }1, \text{Link }1}$
    \end{column}
    \begin{column}{0.44\textwidth}
      \begin{tabular}{r|c}
        Soluzione & Costo totale \\
        \hline
        \textcolor{green}{Ottima} & $0.75$ \\
        \textcolor{blue}{Nash} & $1$
      \end{tabular}
    \end{column}
  \end{columns}
\end{frame}

\begin{frame}{Instradamento ``di Nash''}
  Per poter utilizzare l'equilibrio di Nash abbiamo bisogno di un
  \textbf{insieme finito di giocatori}, quindi dobbiamo
  \textbf{discretizzare il traffico}.

  Ma l'equilibrio di Nash potrebbe non esiste (o non essere unico) e
  potrebbe essere difficile da calcolare (diventa un problema di
  programmazione intera).
  \vfill

  Concentriamoci su problemi con una sorgente e una destinazione in
  cui il traffico è formato da unità discrete omogenee.
\end{frame}

\begin{frame}{Instradamento ``di Nash'': cos'è?}
  Nel nostro caso i giocatori sono dati dalle unità di traffico e
  l'insieme delle scelte è $P$, cioè l'insieme dei percorsi da
  sorgente a destinazione.
  \vfill
  
  Ricordiamo che $p^* \in P^{\mathcal{I}}$ è un equilibrio di Nash se
  \[ \forall i\in \mathcal{I}\;\; \forall p_i \in P \;\;\;
    u_i\pa{p^*_i,p^*_{-i}} \le u_i\pa{p_i,p^*_{-i}} \]
  dove $u_i\pa{p} = \sum _{j\in p_i} m\cdot l(x_j)$ è il costo pagato
  dall'utente $i$ per percorrere il percorso $p_i$.

  \begin{myoss}
    Dato $i$ e fissato $p^*_{-i}$ stiamo minimizzando la funzione
    \[ p_i \longmapsto u_i\pa{p_i,p^*_{-i}} \]
  \end{myoss}
\end{frame}

\begin{frame}{Equilibrio per molti giocatori}
  \begin{myoss}
    Dato $i$ e fissato $p^*_{-i}$ stiamo minimizzando la funzione
    \[ p_i \longmapsto u_i\pa{p_i,p^*_{-i}} \]
  \end{myoss}
  Dato $p^*_{-i}$ l'utente $i$ sta scegliendo il percorso meno
  costoso.
  \vfill
  
  Se il flusso di ogni utente è molto piccolo (stiamo pensando ad un
  limite) possiamo quindi dire che il costo dei percorsi scelti sarà
  (circa) uguale per tutti i giocatori e sarà quello minimo.
\end{frame}

\begin{frame}{Equilibrio di Wardrop}
  \begin{mydef}[Equilibrio di Wardrop]
    Una soluzione $x$ di $R\in \mathcal{R}$ è un \textbf{equilibrio di
      Wardrop} se
    \[
      \begin{matrix}
        \sum _{i\in p'} l_i(x_i) = \sum _{i\in p} l_i(x_i) & & \forall
        p,p'\in P \text{ con } x_p >0, x_{p'} >0 \\
        \sum _{i\in p'} l_i(x_i) \ge \sum _{i\in p} l_i(x_i) & & \forall
        p,p'\in P \text{ con } x_p >0, x_{p'} =0 
      \end{matrix}
    \]
    indichiamo con $x^{WE}(R)$ una tale soluzione.
  \end{mydef}

  \begin{myteo}[Realazione tra gli equilibri di Nash e
    Wardrop\footnote{A. Haurie and P. Yarcotte, On the Relationship
      between Nash-Cournot and Wardrop Equilibria, NETWORKS Vol 15(1985) 295-308, teoremi 3.1 e 3.2}]
    Sotto opportune ipotesi, data una rete $R$ e delle sue approssimazioni
    discrete $R_n$, se per ogni $n$ esiste un equilibrio di Nash allora
    l'equilibrio di Wardrop di $R$ coincide con il limite degli equilibri di
    Nash delle $R_n$.
  \end{myteo}
\end{frame}

\subsection{Il prezzo dell'anarchia (POA)}


\section{Instradamento parzialmente ottimo}

\subsection{Provider e sottoreti}

\subsection{Efficienza e limiti}


\section{Concorrenza sui costi}

\subsection{Modello}

\subsection{Monopoli e oligopoli}






\end{document}

