\documentclass[a4paper]{article}
%\documentclass[a4paper,10pt]{article}

\usepackage{amsmath}
\usepackage{amssymb}
\usepackage{amsthm}
% \usepackage{xfrac}
% \usepackage[all]{xy}
\usepackage{mathtools}
\usepackage{graphicx}
% \usepackage{fullpage}
\usepackage{hyperref}
\usepackage[utf8x]{inputenc}
\usepackage[italian]{babel}
%\usepackage{lmodern}

% \usepackage{pdftricks}
% \begin{psinputs}
%    \usepackage{pstricks}
%    \usepackage{multido}
% \end{psinputs}

\usepackage{ulem}

\usepackage{tikz}
\usetikzlibrary{arrows}

% \setlength{\parindent}{0in}

\newcounter{counter1}

\theoremstyle{plain}
\newtheorem{myteo}[counter1]{Teorema}
\newtheorem{mylem}[counter1]{Lemma}
\newtheorem{mypro}[counter1]{Proposizione}
\newtheorem{mycor}[counter1]{Corollario}
\newtheorem*{myteo*}{Teorema}
\newtheorem*{mylem*}{Lemma}
\newtheorem*{mypro*}{Proposizione}
\newtheorem*{mycor*}{Corollario}

\theoremstyle{definition}
\newtheorem{mydef}[counter1]{Definizione}
\newtheorem{myes}[counter1]{Esempio}
\newtheorem{myex}[counter1]{Esercizio}
\newtheorem*{mydef*}{Definizione}
\newtheorem*{myes*}{Esempio}
\newtheorem*{myex*}{Esercizio}

\theoremstyle{remark}
\newtheorem{mynot}[counter1]{Nota}
\newtheorem{myoss}[counter1]{Osservazione}
\newtheorem*{mynot*}{Nota}
\newtheorem*{myoss*}{Osservazione}

\newcommand{\obar}[1]{\overline{#1}}
\newcommand{\ubar}[1]{\underline{#1}}

\newcommand{\set}[1]{\left\{#1\right\}}
\newcommand{\pa}[1]{\left(#1\right)}
\newcommand{\ang}[1]{\left<#1\right>}
\newcommand{\bra}[1]{\left[#1\right]}
\newcommand{\abs}[1]{\left|#1\right|}
\newcommand{\norm}[1]{\left\|#1\right\|}

\newcommand{\pfrac}[2]{\pa{\frac{#1}{#2}}}
\newcommand{\bfrac}[2]{\bra{\frac{#1}{#2}}}
\newcommand{\psfrac}[2]{\pa{\sfrac{#1}{#2}}}
\newcommand{\bsfrac}[2]{\bra{\sfrac{#1}{#2}}}

\newcommand{\der}[2]{\frac{\partial #1}{\partial #2}}
\newcommand{\pder}[2]{\pfrac{\partial #1}{\partial #2}}
\newcommand{\sder}[2]{\sfrac{\partial #1}{\partial #2}}
\newcommand{\psder}[2]{\psfrac{\partial #1}{\partial #2}}

\newcommand{\intl}{\int \limits}

\DeclareMathOperator{\de}{d}
\DeclareMathOperator{\id}{Id}
\DeclareMathOperator{\len}{len}

\DeclareMathOperator{\gl}{GL}
\DeclareMathOperator{\aff}{Aff}
\DeclareMathOperator{\isom}{Isom}

\title{Problemi di instradamento su reti}
\date{\today}
\author{Enrico Polesel}

\begin{document}
\maketitle

\section{Introduzione ed esempi}

Nello studio dei trasporti e delle reti di telecomunicazioni è
importante riuscire a prevedere il comportamento degli utenti e la
distribuzione del loro traffico in funzione dello stato della rete e
delle decisioni prese dai gestori della rete.

Tra i molti esempi di reti a cui potrebbe adattarsi il nostro modello
ne citiamo due molto diffusi: la \textbf{rete stradale} e la
\textbf{rete internet}. L'esperienza ci dice che non tutte le strade
che congiungono due località sono equivalenti: strade diverse potranno
avere tempi di percorrenza diversi (che potranno dipendere dal
traffico presente). Per questo ci servirà introdurre dei costi che,
nei nostri due esempi, si tradurranno in tempi di percorrenza (detti
anche \textit{latenze} nell'ambito della rete internet).

Vedremo alcuni modi in cui potrà disporsi il traffico sulle nostre
reti, il più importante è l'\textbf{equilibrio di Wardrop} che bene
esprime la volontà di ogni utente di minimizzare il proprio costo
arrivando all'equilibrio in cui ``non esistono scorciatoie''.

Infine vedremo anche, se pur in un modello molto semplificato, cosa
succede quando al tempo di percorrenza aggiungiamo anche il costo
economico fissato da uno o più fornitori che tentano di massimizzare
il proprio guadagno.

\section{Modello e instradamento socialmente ottimo}

\subsection{Modello}

Modellizziamo le nostre reti come grafi diretti dove per ogni arco è
data una funzione costo in funzione del traffico che attraversa l'arco
stesso (il costo è quindi una funzione ``locale'').

Supponiamo che le funzioni costo siano continue, non negative e non
decrescenti. Quest'ultima ipotesi sulla monotonia può essere
giustificata dall'idea che le prestazioni di una strada (o un
collegamento) diminuiscono con l'aumentare del traffico.

Dato il traffico che deve attraversare la nostra rete, lo dividiamo in
base alle coppie ``sorgente-destinazione'' e indicizziamo queste
coppie con un qualche insieme $W$, descriviamo il traffico con tre
vettori: vettore delle sorgenti, vettore delle destinazioni e vettore
delle quantità di traffico da instradare.

Supponiamo tutto il traffico \textbf{infinitamente divisibile}, cioè
possiamo dividerlo sulle nostre strade in rapporti arbitrari. Questo
confligge con la realtà delle reti stradali e internet in cui il
traffico è discreto, ma ne descrive bene il comportamento quando la
quantità di utenti è sufficientemente grande.

Infine supponiamo che ci venga fornito per ogni coppia
``sorgente-destinazione'' l'insieme dei percorsi non contenenti cicli
che le congiungono. Questo ci permette di non dover trattare la
costruzione di questi insiemi (che comunque è un problema
semplice\footnote{\url{https://xlinux.nist.gov/dads/HTML/allSimplePaths.html}}

In figura \ref{fig:esempio-rete} si vede un esempio di rete.

\begin{figure}[h]
  \begin{tikzpicture}[mnode/.style={circle,fill=blue!20},>=latex]
    \node[mnode] (1) at (0,0) {1};
    \node[mnode] (2) at (2,2) {2};
    \node[mnode] (3) at (4,2) {3};
    \node[mnode] (4) at (3,-2) {4};
    \node[mnode] (5) at (6,0) {5};
    \draw [->] (1) to [bend left] (2);
    \node at (1,1) {$l_1(x_1)$};
    \draw [->] (2) to [bend right] (3);
    \node at (3,2.7) {$l_2(x_2)$};
    \draw [->] (3) to [bend right] (2);
    \node at (3,1.3) {$l_3(x_3)$};
    \draw [->] (3) to [bend left] (5);
    \node at (5.5,1.7) {$l_4(x_4)$};
    \draw [->] (1) to [bend right] (4);
    \node at (0.6,-1.7) {$l_5(x_5)$};
    \draw [->] (3) -- (4);
    \node at (4,0) {$l_6(x_6)$};
    \draw [->] (4) -- (2);
    \node at (2,0) {$l_7(x_7)$};
    \draw [->] (4) to [bend right] (5);
    \node at (5,-1) {$l_8(x_8)$};

    \node (S1) at (-2,0) {};
    \node (S2) at (-1,2) {};
    \node (T1) at (8,0) {};
    \node (T2) at (5.5,-2) {};
    \draw[->] (S1) -- node[above]{traffico 1} (1);
    \draw[->] (5) -- node[above]{traffico 1} (T1);
    \draw[->] (S2) -- node[above]{traffico 2} (2);
    \draw[->] (4) -- node[below]{traffico 2} (T2);
  \end{tikzpicture}
  \caption{Esempio di rete e traffico}
  \label{fig:esempio-rete}
\end{figure}

Formalizzando quanto appena detto, un'instanza del nostro problema è
data dalla seguente definizione:

\begin{mydef}[Instanza di instradamento]
  Un'\textbf{istanza di instradamento} (\textit{routing instance} o
  più brevemente \textbf{rete}) è $R=(V,A,P,s,t,X,l)$ con:
  \begin{itemize}
  \item $(V,A)$ il grafo diretto,
  \item $s = \bra{s_i}_{i\in W}$ vettore delle sorgenti (per un opportuno $W$),
  \item $t = \bra{t_i}_{i\in W}$ vettore delle destinazioni,
  \item $P = \cup _{i\in W} P_i$ con $P_i$ insieme dei
    percorsi da $s_i$ a $t_i$,
  \item $X = \bra{X_i}_{i\in W}$ dove $X_i$ rappresenta la quantità di
    traffico da instradare da $s_i$ a $t_i$,
  \item $l = \bra{l_j}_{j\in A}$ dove $l_j(x_j)$ è il costo del link
    $j$ con un traffico $x_j$.
  \end{itemize}
  Supponiamo che le funzioni $l_j$ siano \textbf{non negative},
  \textbf{continue} e \textbf{non decrescenti}.
\end{mydef}

TODO: $\mathcal{R}$.

Per descrivere un possibile schema secondo cui il traffico può
disporsi utilizziamo due vettori: $\bra{x_j}_{j\in A}$ per descrivere
il traffico su ogni arco e $\bra{x_p}_{p\in P}$ per descrivere il
traffico su ogni percorso.

Questi due vettori (che esprimiamo come un unico vettore $x$)
descrivono un possibile schema se soddisfano la definzione di soluzione:

\begin{mydef}[Soluzione di un'instanza]
  Data un'istanza $R=(V,A,P,s,t,X,l)$ un vettore
  $x = \bra{x_j}_{j\in A} \cup \bra{x_p}_{p\in P}$ è \textbf{soluzione} se:
  \begin{itemize}
  \item instradiamo solo quantità non negative di traffico: \[ \forall p\in P\;\; x_p \ge 0\]
  \item il traffico su ogni arco è dato dalla somma del traffico di
    ogni percorso che lo attraversa: \[ \forall j\in A\;\; x_j = \sum_{p\in P,p\ni j} x_p\]
  \item la quantità di traffico instradata tra ogni coppia è proprio
    quella richiesta dal problema: \[\forall i\in W\;\; \sum_{p\in P_i} x_p = X_i\]
  \end{itemize}
\end{mydef}




\section{Instradamento di Wardrop}


\begin{myteo}
  Il costo globale dell'equilibrio di Wardrop è unico (TODO)
\end{myteo}
\begin{proof}
  Abbiamo visto che $x$ è un equilibrio di Wardrop se e solo se è
  minimo (vincolato) della funzione $f(x) = \sum _{j\in A} \int _0
  ^{x_j} l_j(z)\de z$, abbiamo anche visto che la funzione $f$ è
  convessa.

  Sia $L_j(x_j) = \int _0 ^{x_j} l_j(z)\de z$, queste sono tutte
  funzioni convesse per le ipotesi sulle funzioni $l_j$.

  Siano $x,y$ due punti di minimo vincolati di $f$, per la convessità
  sappiamo che per ogni $t\in\bra{0,1}$ si ha
  \[ f\pa{ tx + (1-t)y} \le tf(x) + (1-t)f(y) \]
  ma in realtà questa relazione vale con l'uguaglianza perché
  $f(x)=f(y)$ è un punto di minimo.

  Ricordando che anche le $L_j$ sono convesse possiamo scrivere
  \begin{align*}
    f\pa{ tx + (1-t)y} & = \sum _{j\in A} L_j\pa{tx_j + (1-t)y_j} \\
    & \le \sum _{j\in A} \pa{ tL_j(x_j) + (1-t)L_j(y_j)} \\
    & = t \sum _{j\in A} L_j(x_j) + (1-t)\sum _{j\in A} L_j(y_j) \\
    & = tf(x) + (1-t)f(y)
  \end{align*}
  ricordando l'equazione precedente si può dedurre che per ogni $j\in
  A$ si ha
  \[ L_j\pa{tx_j + (1-t)y_j} = tL_j(x_j) + (1-t)L_j(y_j) \]
  quindi le $L_j$ sono lineari in $\bra{x_j,y_j}$, ma questo accade se
  e solo se le $l_j$ sono costanti negli stessi intervalli.

  Quindi per ogni $j\in A$ abbiamo $l_j(x) = l_j(y)$ e quindi $C(x) = C(y)$.  
\end{proof}

\section{Instradamento parzialmente ottimo}

\section{Instradamento in presenza di costi fissati dai gestori della rete}

\appendix

\section{Teoria dei giochi}

\cite{menache2011network}

\bibliographystyle{alpha}

\bibliography{menache2011network}

\end{document}