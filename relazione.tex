\documentclass[a4paper]{article}
%\documentclass[a4paper,10pt]{article}

\usepackage{amsmath}
\usepackage{amssymb}
\usepackage{amsthm}
% \usepackage{xfrac}
% \usepackage[all]{xy}
\usepackage{mathtools}
\usepackage{graphicx}
% \usepackage{fullpage}
\usepackage{hyperref}
\usepackage[utf8x]{inputenc}
\usepackage[italian]{babel}
%\usepackage{lmodern}

% \usepackage{pdftricks}
% \begin{psinputs}
%    \usepackage{pstricks}
%    \usepackage{multido}
% \end{psinputs}

\usepackage{ulem}

\usepackage{tikz}
\usetikzlibrary{arrows}

% \setlength{\parindent}{0in}

\newcounter{counter1}

\theoremstyle{plain}
\newtheorem{myteo}[counter1]{Teorema}
\newtheorem{mylem}[counter1]{Lemma}
\newtheorem{mypro}[counter1]{Proposizione}
\newtheorem{mycor}[counter1]{Corollario}
\newtheorem*{myteo*}{Teorema}
\newtheorem*{mylem*}{Lemma}
\newtheorem*{mypro*}{Proposizione}
\newtheorem*{mycor*}{Corollario}

\theoremstyle{definition}
\newtheorem{mydef}[counter1]{Definizione}
\newtheorem{myes}[counter1]{Esempio}
\newtheorem{myex}[counter1]{Esercizio}
\newtheorem*{mydef*}{Definizione}
\newtheorem*{myes*}{Esempio}
\newtheorem*{myex*}{Esercizio}

\theoremstyle{remark}
\newtheorem{mynot}[counter1]{Nota}
\newtheorem{myoss}[counter1]{Osservazione}
\newtheorem*{mynot*}{Nota}
\newtheorem*{myoss*}{Osservazione}

\newcommand{\obar}[1]{\overline{#1}}
\newcommand{\ubar}[1]{\underline{#1}}

\newcommand{\set}[1]{\left\{#1\right\}}
\newcommand{\pa}[1]{\left(#1\right)}
\newcommand{\ang}[1]{\left<#1\right>}
\newcommand{\bra}[1]{\left[#1\right]}
\newcommand{\abs}[1]{\left|#1\right|}
\newcommand{\norm}[1]{\left\|#1\right\|}

\newcommand{\pfrac}[2]{\pa{\frac{#1}{#2}}}
\newcommand{\bfrac}[2]{\bra{\frac{#1}{#2}}}
\newcommand{\psfrac}[2]{\pa{\sfrac{#1}{#2}}}
\newcommand{\bsfrac}[2]{\bra{\sfrac{#1}{#2}}}

\newcommand{\der}[2]{\frac{\partial #1}{\partial #2}}
\newcommand{\pder}[2]{\pfrac{\partial #1}{\partial #2}}
\newcommand{\sder}[2]{\sfrac{\partial #1}{\partial #2}}
\newcommand{\psder}[2]{\psfrac{\partial #1}{\partial #2}}

\newcommand{\intl}{\int \limits}

\DeclareMathOperator{\de}{d}
\DeclareMathOperator{\id}{Id}
\DeclareMathOperator{\len}{len}

\DeclareMathOperator{\gl}{GL}
\DeclareMathOperator{\aff}{Aff}
\DeclareMathOperator{\isom}{Isom}

\title{Problemi di instradamento su reti}
\date{\today}
\author{Enrico Polesel}

\begin{document}
\maketitle

\section{Introduzione ed esempi}

Nello studio dei trasporti e delle reti di telecomunicazioni è
importante riuscire a prevedere il comportamento degli utenti e la
distribuzione del loro traffico in funzione dello stato della rete e
delle decisioni prese dai gestori della rete.

Tra i molti esempi di reti a cui potrebbe adattarsi il nostro modello
ne citiamo due molto diffusi: la \textbf{rete stradale} e la
\textbf{rete internet}. L'esperienza ci dice che non tutte le strade
che congiungono due località sono equivalenti: strade diverse potranno
avere tempi di percorrenza diversi (che potranno dipendere dal
traffico presente). Per questo ci servirà introdurre dei costi che,
nei nostri due esempi, si tradurranno in tempi di percorrenza (detti
anche \textit{latenze} nell'ambito della rete internet).

Vedremo alcuni modi in cui potrà disporsi il traffico sulle nostre
reti, il più importante è l'\textbf{equilibrio di Wardrop} che bene
esprime la volontà di ogni utente di minimizzare il proprio costo
arrivando all'equilibrio in cui ``non esistono scorciatoie''.

Infine vedremo anche, se pur in un modello molto semplificato, cosa
succede quando al tempo di percorrenza aggiungiamo anche il costo
economico fissato da uno o più fornitori che tentano di massimizzare
il proprio guadagno.

\section{Modello e instradamento socialmente ottimo}

\section{Instradamento di Wardrop}

\section{Instradamento parzialmente ottimo}

\section{Instradamento in presenza di costi fissati dai gestori della rete}

\appendix

\section{Teoria dei giochi}

\cite{menache2011network}

\bibliographystyle{alpha}

\bibliography{menache2011network}

\end{document}